% !TEX TS-program = xelatex
% !TEX encoding = UTF-8 Unicode
% !Mode:: "TeX:UTF-8"

\documentclass{resume}
\usepackage{zh_CN-Adobefonts_external} % Simplified Chinese Support using external fonts (./fonts/zh_CN-Adobe/)
%\usepackage{zh_CN-Adobefonts_internal} % Simplified Chinese Support using system fonts
\usepackage{linespacing_fix} % disable extra space before next section
\usepackage{cite}

\begin{document}
\pagenumbering{gobble} % suppress displaying page number

\name{李浩天}

% {E-mail}{mobilephone}{homepage}
% be careful of _ in emaill address
% \contactInfo{18317713768}{形式化分析}{lihaotian@bupt.edu.cn}{}
\contactInfo{18317713768}{后端开发工程师}{lihaotian@bupt.edu.cn}{}
% {E-mail}{mobilephone}
% keep the last empty braces!
%\contactInfo{xxx@yuanbin.me}{(+86) 131-221-87xxx}{}
 
% \section{个人总结}
% 本人在校成绩优秀、乐观向上,工作负责、自我驱动力强、热爱尝试新事物,认同开放、连接、共享的Web在未来的不可替代性。在校期间长期从事可视分析(Web的2D/3D时空可视化)相关研究,对Web技术发展趋势及前端工程化解决方案有浓厚兴趣。\textbf{现任职于 BAT 集团。}

% \section{\faGraduationCap\ 教育背景}
\section{教育背景}
\datedsubsection{\textbf{北京邮电大学},网络与信息安全,\textit{在读硕士研究生}}{2024.9 - 至今}
\ \textbf{主修课程:}信息安全实验,无线通信安全,密码学
\datedsubsection{\textbf{河南科技大学},物联网工程,\textit{工学学士}}{2020.9 - 2024.6}
\ \textbf{主修课程:}操作系统,数据结构,计算机网络

% \section{\faCogs\ IT 技能}
\section{技术能力}
% increase linespacing [parsep=0.5ex]
\begin{itemize}[parsep=0.2ex]
  \item \textbf{计算机基础}: 熟练掌握操作系统,计算机网络,数据结构和算法
  \item \textbf{框架}:熟练掌握 Spring,SpringBoot,SpringMVC,MyBatis 开发框架
  \item \textbf{数据库}:熟练使用 MySQL,Redis
  % \item \textbf{形式化分析工具}: 熟练掌握使用Tamarin-prover、DeepSec工具
\end{itemize}

% \end{itemize}
\section{实习经历}
\datedsubsection{\textbf{约翰威立商务服务(北京)有限公司}}{2025.07.16-2025.09.30}
项目背景是为用户在微信端提供订阅期刊状态更新的通知,支持用户订阅期刊,支持订阅的期刊状态发生变化时,发送通知给微信
\begin {itemize}
\item 基于 Rex 系统 API 开发用户文章状态查询功能,实现查询结果的数据库持久化存储,支撑文章状态数据的系统化管理
\item 设计 Kafka 消息消费架构处理用户文章状态变更(涵盖 update/delete 事件):通过抽象类封装公共处理流程(消息过滤→KafkaEvent对象封装→消费幂等性校验→数据抽取→持久化存储)
\item 实现微信通知定时发送机制,配置特定间隔的周期性推送任务
\item 构建通知失败重传机制,设置三次重试阈值,超出后自动触发开发者邮件告警,保障通知链路可靠性
\item 开发用户订阅功能模块,支持用户自主选择关注内容,完善服务交互体系
\end {itemize}
\datedsubsection{\textbf{美团}}{2025.11.27-至今}
需求背景是民宿业务线准备引入订单通作为商家曝光的套餐,需要在用户下单阶段准确识别订单来源,
判断订单是否由订单通带来,以保障平台订单通相关收入的准确统计。
主要负责在归因服务中引入民宿订单通完整业务链路,负责站外订单通的归因判定逻辑设计与实现
\begin {itemize}
\item 基于交易侧请求进行参数校验与过滤,确保归因流程仅在有效订单场景下触发
\item 结合商家账户与房源信息,校验商家是否具备订单通投放费用,提前终止无效归因请求
\item 根据用户下单位置与商家位置进行距离判断,减少不必要的归因计算
\item 设计并实现归因链处理流程,包括数据补齐、归因匹配、结果下发、归因重写、渠道映射订单通等关键步骤
\end {itemize}
\section{项目经历}
\datedsubsection{\textbf{美食点评项目}}{2024.09-2024.12}
\begin{itemize}
  \item 基于 JWT 实现用户身份认证机制,保障接口访问安全性
  \item 设计前端流量控制策略,通过按钮置灰、图形验证码及排队队列限制请求频次,有效过滤无效请求
  \item 引入 RabbitMQ 实现订单流程异步化处理,解耦业务链路,提升系统吞吐量
  \item 使用Redis缓存热点数据,减轻数据库压力
  \item 结合乐观锁与 Lua 脚本实现库存原子操作,解决高并发下商品超卖问题,确保数据一致性
  \item 使用分布式锁,解决一人一单问题
\end{itemize}
% \datedsubsection{\textbf{面向6G的移动通信安全研究}}{2024-2025}
% \begin{itemize}
%   \item 对5G-AKA协议进行学习
%   \item 使用Tamarin-prover对5G-AKA协议进行形式化建模
%   \item 对5G-AKA协议进行改进,并使用Tamarin-prover获得认证性、机密性的最小安全假设
%   \item 使用USRP-B210, Free5gc, SrsUE, SrsRAN对改进协议进行实现
% \end{itemize}
\section{个人总结}
% increase linespacing [parsep=0.5ex]
\begin{itemize}[parsep=0.2ex]
 \item \textbf{保研至北京邮电大学},学业基础扎实且目标明确
 \item \textbf{以学生第一作者身份发表CCF A 类论文1篇},发表在网安四大顶会之一,ACM CCS,录用率仅13\%
 \item  \textbf{获得研究生国家奖学金},具备优秀的学习能力与自我驱动力
\item  获得北京邮电大学研究生一等学业奖学金
 \item  通过英语四六级
%  \item 学习与实践转化能力强,曾独立在\textbf{1个月内}完成实验平台的搭建、调试与测试,保障项目高效推进。
\end{itemize}
% \section{荣誉证书}
% % increase linespacing [parsep=0.5ex]
% \begin{itemize}[parsep=0.2ex]
% %   \item LeetCodeOJ Solutions, \textit{https://github.com/hijiangtao/LeetCodeOJ}
%   \item \textit{<5G-RNAKA: A Random Number-based Authentication and Key
% Agreement Protocol for 5G Systems>}已录用在ACM CCS 2025,网安四大顶会,CCF A类,学生一作
%   \item 硕士期间获得国家奖学金  
%   \item 硕士期间获得北京邮电大学一等奖学金
  % \item 本科期间获得国家励志奖学金
  % \item 中国互联网+大赛省级二等奖 
  % \item 通过英语四六级
  % \item 通过计算机二级等级考试 (java)
%   \item 中国机器人大赛暨Robocup公开赛(武术擂台赛)全国一等奖,2013年10月
  % \item 第11届北京理工大学“世纪杯”竞赛学生课外科技作品竞赛\textbf{特等奖},2013年8月
  % \item VIS Components for security system, \textit{https://hijiangtao.github.io/ss-vis-component/}
  % \item 个人博客:\textit{https://hijiangtao.github.io/},更多作品见 \textit{https://github.com/hijiangtao}
%   \item 电视节目"爸爸去哪儿"可视化分析展示, \textit{https://hijiangtao.github.io/variety-show-hot-spot-vis/}
% \end{itemize}

% \section{\faHeartO\ 项目/作品摘要}
% \section{项目/作品摘要}
% \datedline{\textit{An Integrated Version of Security Monitor Vis System}, https://hijiangtao.github.io/ss-vis-component/ }{}
% \datedline{\textit{Dark-Tech}, https://github.com/hijiangtao/dark-tech/ }{}
% \datedline{\textit{融合社交网络数据挖掘的电视节目可视化分析系统}, https://hijiangtao.github.io/variety-show-hot-spot-vis/}{}
% \datedline{\textit{LeetCodeOJ Solutions}, https://github.com/hijiangtao/LeetCodeOJ}{}
% \datedline{\textit{Info-Vis}, https://github.com/ISCAS-VIS/infovis-ucas}{}


% \section{\faInfo\ 社会实践/其他}
% \section{社区参与/实践其他}
% % increase linespacing [parsep=0.5ex]
% \begin{itemize}[parsep=0.2ex]
%   \item 乐于参与开源社区讨论,\textbf{参与翻译 Vue.js, webpack, WebAssembly, Babel 文档,印记中文成员}
%   \item 中国科学院大学2016秋季学期可视化与可视分析课程助教,\textit{http://vis.ios.ac.cn/infovis-ucas/}
%   \item 未来论坛学生会成员、北理社联新闻信息中心主任、北理工软件学院学生会宣传部副部长(2012-2016)
%   \item 2013-2015 北京市共青团“温暖衣冬”志愿者,第九届园博会志愿者,2014 FLL机器人世锦赛志愿者
% \end{itemize}

%% Reference
%\newpage
%\bibliographystyle{IEEETran}
%\bibliography{mycite}
% 订单来源可能来自站内或站外渠道,若仅依赖站内标识,容易导致站外订单通订单漏记或错记。
% 因此在订单创建过程中,需要在无法确认站内订单通来源时,引入归因服务,对订单是否为站外订单通带来进行判定。
\end{document}

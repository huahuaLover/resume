% !TEX TS-program = xelatex
% !TEX encoding = UTF-8 Unicode
% !Mode:: "TeX:UTF-8"

\documentclass{resume}
\usepackage{zh_CN-Adobefonts_external} % Simplified Chinese Support using external fonts (./fonts/zh_CN-Adobe/)
%\usepackage{zh_CN-Adobefonts_internal} % Simplified Chinese Support using system fonts
\usepackage{linespacing_fix} % disable extra space before next section
\usepackage{cite}

\begin{document}
\pagenumbering{gobble} % suppress displaying page number

\name{李浩天}

% {E-mail}{mobilephone}{homepage}
% be careful of _ in emaill address
\contactInfo{(+86) 18317713768}{Java后端工程师}{lihaotian@bupt.edu.cn}{}
% {E-mail}{mobilephone}
% keep the last empty braces!
%\contactInfo{xxx@yuanbin.me}{(+86) 131-221-87xxx}{}
 
% \section{个人总结}
% 本人在校成绩优秀、乐观向上,工作负责、自我驱动力强、热爱尝试新事物,认同开放、连接、共享的Web在未来的不可替代性。在校期间长期从事可视分析(Web的2D/3D时空可视化)相关研究,对Web技术发展趋势及前端工程化解决方案有浓厚兴趣。\textbf{现任职于 BAT 集团。}

% \section{\faGraduationCap\ 教育背景}
\section{教育背景}
\datedsubsection{\textbf{北京邮电大学},网络与信息安全,\textit{在读硕士研究生}}{2024.9 - 至今}
\ \textbf{主修课程:}信息安全实验,无线通信安全,密码学
\datedsubsection{\textbf{河南科技大学},物联网工程,\textit{工学学士}}{2020.9 - 2024.6}
\ \textbf{主修课程:}操作系统,数据结构,计算机网络

% \section{\faCogs\ IT 技能}
\section{技术能力}
% increase linespacing [parsep=0.5ex]
\begin{itemize}[parsep=0.2ex]
  \item \textbf{计算机基础}: 熟练掌握计算机网络,数据结构和算法,操作系统,
  \item \textbf{框架}: 熟练掌握Spring,SpringBoot,SpringMVC,MyBatis开发框架
  \item \textbf{数据库}: 熟练使用MySQL,Redis及常见优化手段
\end{itemize}

% \end{itemize}

\section{项目经历}
\datedsubsection{\textbf{道路交通管理系统}}{2022-2023}
\begin{itemize}
  \item 采用Echarts、Vue、Element-UI设计前端页面
  \item 采用SpringBoot、MyBatis、MySql构建后端接口
  \item 百度API实现车牌识别、高德API实现道路规划和实时道路推送、阿里云实现短信验证
\end{itemize}
\datedsubsection{\textbf{美食点评项目}}{2024-2025}
\begin{itemize}
  \item 采用 Redis 实现Session共享登录
  \item 采用 Redis 实现秒杀的高并发
  \item 采用 Redis 实现隐藏秒杀接口,接口限流的功能
  \item 采用 Rabit-MQ 的Topic模式实现异步消息处理,提高吞吐量
  \item 采用 Redis 实现商户信息缓存策略,解决缓存穿透、缓存雪崩、缓存击穿等问题,提升数据读取性能
  \item 采用 乐观锁和lua脚本解决商品超卖的问题
  \item 采用 分布式锁解决一人一单的问题
  \item 采用 BitMap 和 GEO 数据结构优化签到统计和商户位置查询
\end{itemize}
% \begin{onehalfspacing}
% \end{onehalfspacing}

% \datedsubsection{\textbf{DID-ACTE} 荷兰莱顿}{2015年}
% \role{本科毕业设计}{LIACS 交换生}
% 利用结巴分词对中国古文进行分词与词性标注,用已有领域知识训练形成 classifier 并对结果进行调优
% \begin{onehalfspacing}
% \begin{itemize}
%   \item 利用结巴分词对中国古文进行分词与词性标注
%   \item 利用已有领域知识训练形成 classifier, 并用分词结果进行测试反馈
%   \item 尝试不同规则,对 classifier 进行调优
% \end{itemize}
% \end{onehalfspacing}
\section{个人总结}
% increase linespacing [parsep=0.5ex]
\begin{itemize}[parsep=0.2ex]
  \item 读书期间,认真努力,年级排名2/72,顺利保研至北京邮电大学。
  \item 善于在工作中提出问题、发现问题、解决问题,有较强的解决问题能力。
  \item 学习能力强,勤奋好学,踏实肯干,通过短时间的学习掌握工作内容并快速上手
\end{itemize}
\section{荣誉证书}
% increase linespacing [parsep=0.5ex]
\begin{itemize}[parsep=0.2ex]
%   \item LeetCodeOJ Solutions, \textit{https://github.com/hijiangtao/LeetCodeOJ}
  \item \textit{<5G-RNAKA: A Random Number-based Authentication and Key
Agreement Protocol for 5G Systems>}已录用在CCS 2025,网安四大顶会,学生一作
  \item 中国互联网+大赛省级二等奖 
  \item 通过英语四六级
  \item 通过计算机二级等级考试(java)
  \item 本科期间获得国家励志奖学金
  \item 硕士期间获得北京邮电大学一等奖学金
%   \item 中国机器人大赛暨Robocup公开赛(武术擂台赛)全国一等奖,2013年10月
  % \item 第11届北京理工大学“世纪杯”竞赛学生课外科技作品竞赛\textbf{特等奖},2013年8月
  % \item VIS Components for security system, \textit{https://hijiangtao.github.io/ss-vis-component/}
  % \item 个人博客:\textit{https://hijiangtao.github.io/},更多作品见 \textit{https://github.com/hijiangtao}
%   \item 电视节目"爸爸去哪儿"可视化分析展示, \textit{https://hijiangtao.github.io/variety-show-hot-spot-vis/}
\end{itemize}

% \section{\faHeartO\ 项目/作品摘要}
% \section{项目/作品摘要}
% \datedline{\textit{An Integrated Version of Security Monitor Vis System}, https://hijiangtao.github.io/ss-vis-component/ }{}
% \datedline{\textit{Dark-Tech}, https://github.com/hijiangtao/dark-tech/ }{}
% \datedline{\textit{融合社交网络数据挖掘的电视节目可视化分析系统}, https://hijiangtao.github.io/variety-show-hot-spot-vis/}{}
% \datedline{\textit{LeetCodeOJ Solutions}, https://github.com/hijiangtao/LeetCodeOJ}{}
% \datedline{\textit{Info-Vis}, https://github.com/ISCAS-VIS/infovis-ucas}{}


% \section{\faInfo\ 社会实践/其他}
% \section{社区参与/实践其他}
% % increase linespacing [parsep=0.5ex]
% \begin{itemize}[parsep=0.2ex]
%   \item 乐于参与开源社区讨论,\textbf{参与翻译 Vue.js, webpack, WebAssembly, Babel 文档,印记中文成员}
%   \item 中国科学院大学2016秋季学期可视化与可视分析课程助教,\textit{http://vis.ios.ac.cn/infovis-ucas/}
%   \item 未来论坛学生会成员、北理社联新闻信息中心主任、北理工软件学院学生会宣传部副部长(2012-2016)
%   \item 2013-2015 北京市共青团“温暖衣冬”志愿者,第九届园博会志愿者,2014 FLL机器人世锦赛志愿者
% \end{itemize}

%% Reference
%\newpage
%\bibliographystyle{IEEETran}
%\bibliography{mycite}
\end{document}
